\documentclass{VUMIFInfKursinis}
\usepackage{algorithmicx}
\usepackage{algorithm}
\usepackage{algpseudocode}
\usepackage{amsfonts}
\usepackage{amsmath}
\usepackage{bm}
\usepackage{color}
\usepackage{graphicx}
% \usepackage{hyperref}  % Nuorodų aktyvavimas
\usepackage{url}


% Titulinio aprašas
\university{Vilniaus universitetas}
\faculty{Matematikos ir informatikos fakultetas}
\institute{Informatikos institutas}  % Užkomentavus šią eilutę - institutas neįtraukiamas į titulinį
\department{Informatikos katedra}
\papertype{Kursinis darbas}
% Nenurodytas vienas arba keli iš būtinų atributų: kalba, raktiniai žodžiai, santrauka.
\title{Automatizuotas kriptovaliutų prekybos robotas}
\titleineng{Automated cryptocurrency trading bot}
\status{3 kurso 1 grupės studentas}
\author{Matas Kaminskas}
\supervisor{J. Asist., Dr. Igor Katin}
\date{Vilnius \\ \the\year}

% Nustatymai
% \setmainfont{Palemonas}   % Pakeisti teksto šriftą į Palemonas (turi būti įdiegtas sistemoje)
\bibliography{bibliografija} 

\begin{document}
\maketitle

\tableofcontents

% Įvade apibūdinamas darbo tikslas, temos aktualumas ir siekiami rezultatai. 
\sectionnonum{Įvadas}
Pasaulis vis labiau modernėja, technologijos tampa vis išmanesnės ir našesnės, nei bet kada anksčiau. Šiais laikais turime kontraversiškai pagarsėjusią
kriptovaliutų rinką, kuri yra kaip skaidresnė atsvara standartinei akciju biržai. Po 2007-2008 metų įvykusios finansų krizės,
visai neužilgo - \cite{SatBitcoin}2009 metais, atsirado pirmoji kriptovaliuta - Bitkoinas, decentralizuotą ir nepriklausoma elektroninę valiutą, paremta
blokų grandinės technologija. Dabartiniais apskaičiavimais tinklalapyje coinmarketcap.com, šios rinkos bendra vertė viršija 1 trilijonus JAV dolerių,
nors mažiau nei prieš metus, ši vertė siekė 3 trilijonus JAV dolerių. Tai tik parodo šios rinkos nepastovumą, bei kodėl ši rinką susilaukia
daug abipusių nuomonių. Nors tai 1 trilijonas JAV dolerių, sudaro tik 10\% akcijų biržos vertės, bei daug kas, jog kriptovaliutos yra laikinas dalykas,
galima pamatyti, jog tai sulaukia vis daugiau susidomėjimo, įtakos ir populiarumo mūsų pasaulyje.    

Kriptovaliutų rinkos pagrindinis skirtumas lyginant su tradicine birža yra jog mainai vyksta 24 valandas per parą. Atvirkščiai nei akcijų birža, kur
darbo valandos yra nustatytos institucijos, dažniausiai standartinėmis darbo valandomis vietiniu laiku. Mainams vykstant be sustojimo yra nepraktiška,
bei neįmanoma visą dieną stebėti ir atlikti prasmingus mainus rinkoje. Atsiranda puiki terpė automatizuoti šį procesą - įdarbinti automatizuotą prekybos
robotą, kuris stebėtų ir atliktų prasmingus mainus šioje sferoje.

Šio darbo tikslas yra ištirti kaip galima sukurti automatizuota kriptovaliutų prekybos robotą naudojant standartinius laiko eilutės autoregresinius modelius
sprendimams priimti. Darbe tiriama, kokį autoregresinį modelį galima taikyti: AR, ARMA, ARIMA, ARFIMA, SARIMA ar kt., tiksliausiai nuspėti kriptovaliutos
kainas ateityje. Pagal gautus rezultatus ir tikslumą galima spręsti, koks modelis yra tinkamiausias pagal poreikius robotui naudoti jį automatizuojant.

Tikslui įgyvendinti keliami uždaviniai:
% bullet points
\begin{itemize}
  \item Atlikti mokslinės literatūros analizę kriptovaliutų ir autoregresinių modelių tema;
  \item Autoregresinių modelių parinkimas ir jų paruošimas prognozuoti kainą.
        % swaps these steps? first robot -> ar models 
  \item Sukurti automatizuotą robotą kuris šiuos modelius gali pritaikyti;
  \item Ekspermentiškai ištirti autoregresinius modelius juos taikant robotui.
        % elaborate a little, effectivnees, speed, accuraccy, simplicity or smth else?
  \item Ištirti gautus rezultatus atlikus modelių taikymą;
  \item Apibendrinti tyrimo metu gautus rezultatus ir apibrėžti išvadas;
\end{itemize}


% bakalauriniame darbe palyginti standartinius time series eilutes modelius vs ML/DL/AI?
%Pagrindinėje tiriamojoje dalyje aptariama ir pagrindžiama tyrimo metodika; ARFIMA AR MODELIAI IR T.t.
%pagal atitinkamas darbo dalis, nuosekliai, panaudojant lyginamosios analizės,
%klasifikacijos, sisteminimo metodus bei apibendrinimus, dėstoma sukaupta ir
%išanalizuota medžiaga.
\section{Pagrindinė tiriamoji dalis}
Pagrindinė ir pirmoji kriptovaliuta visoje rinkoje - Bitcoin. Stebint kriptovaliutų rinka matoma, jog nuo šios valiutos kainos koreliuoja visos kitos kriptovaliutų kainos.
Kyla bitcoin kainą - kyla kitų valiutų kainą, krenta bitcoin kainą - krenta kitų valiutų kainą. Retais atvejais kriptovaliuta sparčiau keičia kainą dėl pažangaus vystymosi ar dėl
% SOL auga labai? XRP freeze del teismo? ADA kyla nes pazangi, nors btc stabilus? ir kiti E.g. nezinu ar reikia čia šitą plėstis
esančių vidinių problemų. Dauguma kriptovaliutų yra riboto kiekio, tad jeigu prekybos robotas per laiko tarpą uždirba maža procentą, jų kainą gali keleriopai išaugti
tikrų valiutų (€ ar \$) atžvilgiu. Vienas iš uždavinių suprasti kaip veikia kriptovaliutų rinka, jog būtų galima automatizuoti šį robotą. Kriptovaliutų rinkoje, kaip ir
akcijų biržoje yra sezoniškumo laikotarpis meškų turgus (angl. bear market) ir bulių turgus(angl. bull market), ankstesnis reiškiantis, jog kainos krenta ir pasilieka žemomis,
o pastaris reiškiantis - atvirkščiai. Dar pastebima jog atskirų populiarų asmenų veikla socialiniuose tinkluose turi didelės įtakos valiutų kainai. Bet šitame darbe stengsimes
atsižvelgti tik į diskrečius duomenys kuriuos galima analizuoti naudojant autoregersinius modelius. Populiarus aforizmas statistikoje "Visi modeliai yra neteisingi", kuris
dažnai yra pratęsiamas kaip "Visi modeliai yra neteisingi, bet kai kurie yra naudingi", todėl šiame darba bus bandoma surasti tinkamiausią modelį šiai problemai.

Pagrindinis tyrimas šiame darbe yra tinkamiausio auto-regresyvaus modelio pritaikymas automatizuotam robotui lošėjui, kuris tiksliai ir preciziškai veiktų pagal nustatymus
t.y. išvengtų klaidų ir būtų galimybė palikti atlikiti savo darbą ilgam laikui. Visa šis procesas gali būti atliekamas žmogaus, galima tam tikrą laiką stebėti besikeičiančias
kainas ir atitinkamai elgtis. Tačiau toks darbas yra monotoniškas ir kyla natūrali mintis jį automatizuoti. Automatizavimui reikia taikyti pasirinktą strategiją, kuri spręstų
kada yra tinkamas metas įsigyti kriptovaliutą, o kada parduoti. Šiame darbe tyrinėsime auto-regresyviuosius taikomus laiko eilutėms, labiausiai paplitusius statistikoje
prognozuojant kaip keisis reikšmė pagal dabartines tendencijas ir kitus niuansus.

% Apie kriptovaliutas galima pašnekėti labiau antroje dalyje tiesiog?
Kriptovaliutos yra paremtos blokų grandinės technologija, todėl daugelis jų yra decentralizuoti tinklai. Visa blokų informacija yra viešai prieinama publikai, puikiai žinoma
tarp kokių šalių įvyksta vyksta sandoriai, taip suteikiamas skaidresnis ir patikimesnis būdas publikai turėti savo nepriklausomą rinką. Ypatingas kriptovaliutų bruožas yra tas,
kad jų paprastai neišleidžia jokia centrinė institucija, todėl teoriškai jos nėra apsaugotos nuo vyriausybės kišimosi ar manipuliavimo. Šiais laikais tai yra gana aktuali tema
kadangi vis daugiau žmonių naudojasi technologijomis ir nori nepriklausomybės nuo institucijų išleidžiamų ir manipuliuojami valiutų, infliacijos ir kt. dalykų. Taipogi sukeltos
ekonominės krizės mažina institucijų pasitikėjimą, todėl kriptovaliutos yra alternatyvi valiuta visuomenei. Nors šiais laikais pastebima tendencija, jog kriptovaliutų sandoriai
dominuoja nereguliuojamose rinkose. Vienas pirmųjų ir didžiausios apyvartos sulaukęs nelegalių prekių tinklalapis "Silkroad" kaip atsiskaitymą už prekes naudojo bitcoiną, tai
skatino bitcoin augimą, kas paspartino ir kitų kriptovaliutų progresą.

\subsection{Darbinė aplinka}
Šiam robotui įgyvendinti naudosime Python programavimo kalbą, kadangi ši kalba turi patogų API su dauguma kriptovaliutų rinkų ir patogias bibliotekas autoregresyviems modeliams.
Docker konteinerius, kad ši programa nepriklausomai nuo kompiuterines sistemos. Taip pat turėsime sujungta duomenų
bazę saugoti analizuotems duomenis. 
Citavimo pavyzdžiai: cituojamas vienas šaltinis \cite{PvzStraipsnLt}; cituojami
keli šaltiniai \cite{PvzStraipsnEn, PvzKonfLt, PvzKonfEn, PvzKnygLt, PvzKnygEn,
  PvzElPubLt, PvzElPubEn, PvzMagistrLt, PvzPhdEn}.

\subsubsection{Python}
Python programavimo kalba. Python yra aukšto lygio programavimo kalba. Joje galima sutikti ne viena programavimo paradigmą:
procedurinę, objektinę ir netgi funkcinę. Ši kalba suteikia patogu abstrakcijos lygį šiai užduočiai - automatizuoti robotą lošėją ir panaudoti AR modelius.
Python yra interpretuojama* kalba, todėl ji veikia nuo failo pradžios iki galo "interpetuojant" kiekvieną eilutę iš viršaus į apačia.
Python yra įgyvendinta naudojant C kalbą, todėl egzistuoja optimalesnis python plėtinys - Cython, jeigu programoje pritrūksta našumo. Šiais laikais
python yra viena populiariausių kalbų pasaulyje, dėl savo paprastumo ir paprasto naudojomo naujam vartotojui, bet tikrai yra daugybę savybių
kuriomis prireiktų ne vienus metus suprasti ir prasmingai naudoti programuojant, lengva išmokti, bet sunku įvaldyti. Dėl šios kalbos privalumų minėtų anksčiau
beveik dauguma egzsistuojančių API bibliotekų palaiko python programavimo kalbą.

\subsubsubsection{Python-binance API}
Šiame darbe naudojama "python binance" API. Ji gali būti lengvai instaliuojama vartotojo sistemoje jeigu yra "pip" paleidžiant komanda terminale "pip install python-binance"
, kur pip yra vartotojo turima pip versija. Ji yra populiarausiai binance API turinti netgi 4.7 tūkstančių žvaigždžių GITHUB platformoje\cite{DokTest}, palyginimui populiaurisias github projektas turi
200 tūstančių žvaigždžių, taigi neoficialiam API projektui turėti tiek dėmėsio yra labai didelis pasiekimas. Šis API yra "wrapperis" oficialiam Binance 
API - binance-connector-python (šis API yra labai paprastas ir lengvasvoris, todėl didelio patogumo nėra, tik paprasčiausios užklausos, kurias visvien
reikėtų apdoroti, ką labai padeda padaryt ankščiau minėta wrapper API - python-binance)  

% Straipsnis kuriuo remiasi šitas darbas
\subsubsection{Binance}
Binance yra didžiausia kriptovaliutų birža pagal prekybos apimtį, turinti itin didelį valiutų pasiulą ir patogiai prieinamus duomenis. Ši birža taip pat
suteikia galimybė "lošti" kriptovaliutomis testiniame tinkle. Tinklo paskirtis yra lengvai nuspėjama iš pavadinimo, jame galima atlikti tuos pačius 
% issiplesti ir papasakoti apie rinkoje galimus atlikti veiksmus, LIMIT BUY, LIMIT SELL ir t.t.
veiksmus kaip realioje rinkoje, tik naudojama netikra valiuta kuri tikros vertės neturi. 

% kas yra SPOT market parašyti? https://academy.binance.com/en/articles/a-complete-guide-to-cryptocurrency-trading-for-beginners
\subsubsection{Binance Spot Test Network}
Testinis tinklas yra beveik identiškas tikrajam "Binance" tinklui, kainos yra vienodos kaip ir pagrindiniame tinkle ar pasaulyje, todėl galima analizuoti 
seniausius rinkos duomenis ir juos taikyti reikiame modelyje. Testinis tinklas nėra toks populiarus, tad jame yra ženkliai mažiau prekiaujančių žmonių. 
Testiniame tinkle pąskyra sukuriama naudojant "Github" prisijungimą. Kiekvienas naujas vartotojas turi pradinį balansą susidendantį iš kriptovaliutų
matomų 1 lentelėje.

\begin{table}[H]\footnotesize
  % tablesgenerator.com - converts calculators (e.g. excel) tables to LaTeX
  \centering
  \caption{Pradinis likutis}    % Antraštė įterpiama prieš lentelę
  {\begin{tabular}{|l|c|} \hline
      Kriptovaliuta & Kiekis  \\
      \hline
      BNB           & 1,000   \\
      BTC           & 1       \\
      BUSD          & 10,000  \\
      ETH           & 100     \\
      LTC           & 500     \\
      TRX           & 500,000 \\
      USDT          & 10,000  \\
      XRP           & 50,000  \\
      \hline 
    \end{tabular}}
\end{table}

Šiuo likučiu galima elgtis kaip norima. Pradinis likutis yra pakankamas atlikti prasmingoms transakcijomis testiniame tinkle. Reikėtų pabrėžti, jog testinis
tinklas kas mėnesį laiko yra iš naujo nustatomas ir visas turimas likutis yra konvertuojamas atgal į pradinį likutį, išvalant ankščiau buvusį balansą, tad jeigu
nepavyko praturtėti prekiaujant kriptovaliutomis, vėl įgaunama proga pradėti prekyba, na o sėkmingu atveju visas pelnas yra ištrinamas ir reikia vėl pradėti
nuo nulio.

% padaryti ka reiškią žymėjimas, PADARYTI KODĖL PASIRENKAmE šitas? žiurėjom į marketa, buvo mažas volume kitų porų 
Kadangi tinklas yra testinis, jame prekiauja mažiau žmonių, šiame tyrime apsiribojama prekiauti šiomis poromis rinkoje: i) BTC/BUSD, ii) LTC/BUSD, iii) XRP/BUSD,
iv) BNB/BTC ir v) ETH/BTC


\subsubsection{Docker}
Docker yra gana nauja technologija, išpopuliarėjusi praeitame dešimtmetyje. Tai yra konteinerių technologija, leidžianti sukurti "konteinerius" ir 
juos dislokuoti. Pavyzdys: sukuriame konteinerį kuriame turime savo visas programos dalis 
(šiuo atveju python programos failus ir config failą, taip pat įmanoma turėti
dockerio konteineryje dar vieną docker konteinerį) ir paleidžiame robotą izoliuotoje aplinkoje. 
Ši technologija yra patogi, nes jeigu veikia lokaliai, tai veiks ir kitur naudojant Docker, nes ši technologija
suvienodina programavimo aplinką, bei nenaudoja tiek daug resursų kaip virtuali mašina.

% autoregression crypto currency predict
% ar, ar-abs, arima, arfima, sarima
\section{Autoregresiniai modeliai ir jų skirstymas}
% interpolation vs extrapolation
Dauguma analizės sričių bando nuspėti kintamųjų elgsena bėgant laikui. Dažnu atveju tai yra vienas ar keli izoliuoti kintamieji, kurie yra stebimi nustatytą laiko
tarpą ar surenkant jų istorinius duomenius. Šiuo būdu analitikai bando nuspėti kintamūjų ateities reikšmes. Iš dalies tai ir yra autoregresinis modeliai. 
Autoregresiniuose modeliuose naudojamos laiko eilučių analizė. Laiko eilučių analizės tikslas nuspėti kaip kinta sekama reikšmė bėgant laikui. Dažnai sutinkami laiko eilučių
duomenų rinkimo pavyzdžiai: sekamas paciento širdies pulsas, prekyboje planuojant sandėlio prekių kiekį ar nuspėti akcijų kainas biržų akcijoje.
Reikia pabrėžti, kad toks analizės būdas tik bando nuspėti koks yra labiausiai tikėtinas rezultatas. Toliau autoregresija išsiskaido į šiuos modelius
\subsection{Autoregressive - AR modelis}
Autoregresinis modelis remiasi tik praeities duomenimis, kad nuspėti kintamojo reikšme ateityje, ieškoma ar pastebimas pasikartojantis modelis, kuris padėtų tiksliau nuspėti dydį ateityje.
Šis modelis dar dažnai vadinamas ARp modeliu, nes naudojamas kintamasis "p", nusakantis kiek praeiteis reikšmių iš laiko periodo norima naudoti.
Laikant, kad kintamasis X yra laiko eilutės kintamasis ARp modelio formulė gali atrodyti taip: \[X_{t} = C + \Phi _{1}X_{t-1}+\epsilon_{t} \]


\subsection {MA - Moving average}
Toliau tyrinėjami autoregresyvieji modeliai susideda iš dar vienos dalies - Slankaus vidurkio - MA (angl. Moving average). Slenkančio vidurkio dabartinė
reikšmė tiesiškai priklauso nuo praeitų reikšmių ir dabartines. Žymėjimas MA(q) reiškia q laipsnio slenkamajį vidurkį, kurio formulę atrodo taip:  
\[X_{t} = \mu + \epsilon_{t} + \sum_{i=1}^{q}\theta_{i}  \epsilon_{t-i}\]

\subsection {ARMA}

\subsection {ARIMA}
ARIMA modelis yra paplitęs nuo 1970-ųjų iki šių dienų. Modelio pavadinimo šifravimas pažodžiui: AR - Autoregresinis modelis,
I - Integracija (Diferencijavimas), atsižvelgiama į duomenų tendenciją, MA - (Moving average) slankusis vidurkis. AR ir MA yra atskiri
modeliai, kurie gali būti naudojami paprastesnei laiko eilučiu analizei. Šio modelio privalumas, jog jis sujungia AR ir MA naudojimą kartu su
diferencijavimu, kuris suteikia galimybė giliau analizuoti laiko eilutes.

\subsection {ARFIMA}
ARFIMA modelis šifruojasi taip pat kaip ARIMA modelis, tik su vienu skirtumu - šiame modelyje yra F, reiškianti "Trupmenomis" (angl. Fractionally).
Šis modelis taip pat apibendrina autoregresyvų integruotą slankiojo vidurkio (ARIMA) modelį su sveikaisiais integracijos laipsniais.
Parametrizavimas sujungia autoregresyvų slankiojo vidurkio (ARMA) modelį, kuris yra plačiai naudojamas trumpos atminties procesu, taip 
patampant ilgos atminties procesu.

\subsection {SARIMA}
SARIMA modelis yra labai panašus, tačiau turi vieną papildomą savybę - sezoniškumą. Pastebėjus, jog periodiškai kartojasi rezultatai laiko eilutėse, 
galima taikyti šį modelį. Geras ir paprastas pavyzdys, prekyba šventiniu laikotarpiu - kalėdos. Prekybos centruose, tuo metu padidėja prekybą, tačiau 
kitą mėnesį ženkliai sumažėja. Spartus sumažėjimas nereiškia, jog prekybai yra didėlės problemos ir reikia pokyčių. Tai dažniausiai yra žmonių 
poilsis nuo pirkinių po šventinio laikotarpio. šis modelis pastebi panašius sezoniškai atsitinkančius įvykius, juos atpažįsta ir pritaiko naudojant
ankščiau minėtą ARIMA modelį. 

\subsection {Autoregressive fractionally integrated moving average - SARFIMA}

\subsection{Pirkti valiutą}
Taigi valiutą perkame jeigu gauname signalą pirkti, kitu atveju robotas skanuoja duomenis nustatyta periodą. Pirkimui naudojamas "Moving Average" algoritmas. 
Vartotojas pats gali nustatyti norimą intervala šiam algoritmui, tačiau rekomenduojamas intervalas yra kuo mažesnis, nes kriptovaliutų rinka 
yra nepastovi ir greitai keičiasi 

% autoregression crypto currency predict
%\subsection(Auto regressive - ABS)

\sectionnonum{Išvados}
Taigi matome, jog galima kurti robotą kuris prekiauja remiantis autoregresiniais modeliais.
% Išvadose ir pasiūlymuose, nekartojant atskirų dalių apibendrinimų,
% suformuluojamos svarbiausios darbo išvados, rekomendacijos bei pasiūlymai.

\sectionnonum{Sąvokų apibrėžimai}
\textbf{Kriptovaliuta} - Kriptovaliuta yra skaitmeninė arba virtuali valiuta, kuri yra apsaugota kriptografija, todėl beveik neįmanoma jos padirbti ar išleisti dvigubai.

\textbf{Kriptovaliutų pora} - Kriptovaliutų pora rinkoje yra naudojama prekiaujant. Kriptovaliutos yra susietos poromis, tad norint nusipirkt BTC valiutos pirmiausia reikia surasti 
galimus keitimo variantus, jei egzistuoja pora BTC/BUSD, galima nusipirkti BTC kriptovaliutos uz turimas BUSD valiutas. Dažnu atveju rinkoje pasidėjus ("FIAT") valiuta
prekybos rinką ją konvertuoją i panašių token valiuta kaip BUSD(us -regulated stablecoin), USDT (usd tether) ar BNB (binance coin)

\textbf{API} - Aplikacijų programavimo sąsaja (angl. Application programming interface), tai sistemos suteikiama sąsaja, kuria galima naudotis norint pasiekti tos sistemos
funckionalumą ar apsikeisti duomenimis.

\textbf{Slankusis vidurkis} - Finansų srityje akcijos ar kito prekiaujamo objekto kainos vidurkio tam tikru periodu rodiklis. Vidurkio skaičiavimo priežastis - padėti išlyginti
kainos duomenis nuolat atnaujinant vidutinę kainą.

% TODO: autoregresinis ar autoregresyvus
\textbf{Autoregresinis modelis} - Statistinis modelis yra autoregresinis, jei jis numato būsimas reikšmes pagal praeities reikšmes. Pavyzdžiui, autoregresinis modelis gali
siekti numatyti būsimas akcijų kainas, remiantis ankstesniais rezultatais.

\printbibliography[heading=bibintoc] % Literatūros šaltiniai aprašomi
% bibliografija.bib faile. Šaltinių sąraše nurodoma panaudota literatūra,
% kitokie šaltiniai. Abėcėlės tvarka išdėstoma tik darbe panaudotų (cituotų,
% perfrazuotų ar bent paminėtų) mokslo leidinių, kitokių publikacijų
% bibliografiniai aprašai (šiuo punktu pasirūpina LaTeX). Aprašai pateikiami
% netransliteruoti.

\appendix  % Priedai
% Prieduose gali būti pateikiama pagalbinė, ypač darbo autoriaus savarankiškai
% parengta, medžiaga. Savarankiški priedai gali būti pateikiami kompiuterio
% diskelyje ar kompaktiniame diske. Priedai taip pat vadinami ir numeruojami.
% Tekstas su priedais siejamas nuorodomis (pvz.: \ref{img:mlp}).

\section{Roboto veikimo struktūra}
\begin{figure}[H]
  \centering
  \includegraphics[scale=0.5]{img/DIAGRAM}
  \caption{Paveikslėlio pavyzdys}   % Antraštė įterpiama po paveikslėlio
  \label{img:diagram}
\end{figure}

\end{document}
